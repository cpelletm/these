\documentclass[a4paper,11pt]{report}
\usepackage[]{amsmath}
\usepackage[]{physics} % \bra, \ket etc
\usepackage{graphicx} %Pour les figures je crois
\usepackage{hyperref}
\usepackage[
    backend=biber, 
    natbib=true,
    style=numeric-comp,
    sorting=none, %Pour faire apparaitre les refs dans l'ordre
    hyperref=true
]{biblatex} %Imports biblatex package
\addbibresource{Bib_intro.bib} %Import the bibliography file

\usepackage{amssymb} %quelques symboles dont gtrsim /lesssim
\usepackage{subcaption} % package pour faire des subfigures
\usepackage{multirow} % package pour multirow/multicolumn
\usepackage{booktabs} % package pour top/mid/bottom rule
\usepackage{tcolorbox} % toujours plus de boites
\usepackage{xcolor} % Pour avoir des couleurs dans les équations

\DeclareUnicodeCharacter{0308}{HERE!HERE!} % je vais pas m'énerver, mais un peu quand meme
\title{}
\begin{document}
\chapter*{Introduction}

Quantum mechanics was first developed at the beginning of the 20th century as an answer to  . In the second half of the 20th century, technological development based on the new understanding of light and matter have led to what was retroactively called the first quantum revolution \citep{thew2019focus}. Technological developments in the 50s and 60s include the semiconductor transistor, the laser, and nuclear magnetic resonance  which have since revolutionized the fields of information science, global communications and medical imaging. 

A new field of quantum technology has emerged in the last two decades. This ``second quantum revolution" differs from the first one in the usage of individual quantum systems, such as single photons, atoms or electrons \citep{peil1999observing}, which have properties that can not be emulated by larger classical systems. These properties include quantum entanglement, superposition or measurement and form the basis of the newly formed quantum information science \citep{nielsen2002quantum, vedral2006introduction, hayashi2006quantum}. 

Central to quantum information science is the idea of quantum bits or qubits \citep{schumacher1996sending}, which are the building blocks of quantum information in analogy to the bits of classical information theory. Physical implementation of a qubit could be any quantum system with two well defined quantum states, as long as these states can be initialized, manipulated and readout \citep{divincenzo2000physical}. Popular qubit candidates include superconducting circuits \citep{nakamura1999coherent, orlando1999superconducting}, photons \citep{bennett1992quantum, zhong2020quantum}, quantum dots \citep{veldhorst2014addressable, zajac2018resonantly}, trapped ions \citep{friis2018observation, wright2019benchmarking} and single crystal defects \citep{jelezko2006single, baranov2011silicon, zhong2015optically}.

Defects in large band gap crystals are sometimes referred to as artificial atoms \citep{buluta2011natural}, as the localized wave function of the electrons surrounding the impurity  give rise to discrete energy levels similar to that of an atom. Of all the possible crystals and defects, the most studied defect for quantum applications is the negatively charged nitrogen-vacancy center in diamond \citep{aharonovich2016solid, de2021materials}, often abbreviated to ``NV center". Although this particular defect of diamond had been known and correctly identified almost 50 years ago \citep{davies1976optical}, interest in the NV center really sparked after it was first isolated in 1997 \citep{gruber1997scanning}. The detection of single NV centers proved that NV centers fluoresce brightly \citep{gruber1997scanning}, are photostable \citep{kurtsiefer2000stable}, and that their electronic spin can be polarized and readout optically \citep{jelezko2004observation}, making them an promising solid state qubit.

Since then, single NV centers have been successfully used to create quantum entanglement over several km \citep{hensen2015loophole}, to store quantum information at room temperature for more than one second \citep{maurer2012room}, to detect single proteins via nuclear magnetic resonance \citep{lovchinsky2016nuclear}, and for many other applications in quantum communication \citep{wehner2018quantum}, quantum computing \citep{de2021materials} or quantum sensing \citep{degen2017quantum}.

As the interest in single NV center grew, diamond synthesis improved significantly \citep{achard2020chemical, barry2020sensitivity, edmonds2020generation}  which benefited not only single NV centers but also NV center ensemble. While single NV center are required for most quantum information processes \citep{ladd2010quantum}, several other applications benefit from a greater concentration of defects and are therefore more suited to operate with NV centers ensemble. Some of these applications are pictured in Fig.

The first application for NV center ensemble is the use of nano or micro diamonds as fluorescent biomarkers. 
   
While most of the NV ensemble experiments work under the assumption that the ensemble is constituted of $N$ independent single NV centers 



\printbibliography
\end{document}