\documentclass[a4paper]{article}
\usepackage[french]{babel}
\usepackage[utf8]{inputenc}
\usepackage[]{amsmath}
\usepackage[]{braket} % \bra, \ket etc
\usepackage{graphicx}




\begin{document}
	While the optimal density of NV centers is generally thought in term of a compromise between sensitivity and  the spins coherent time $T_2^*$, other more exotic effects start to appear when the spin density reaches a critical point. 
	
	For concentration higher than $\sim$ 1~ppm (corresponding to a mean distance between spins of $\approx$ 15~nm or a dipolar coupling strength of $\approx$ 15~KHz), the dipolar coupling between the NV spins starts to play an important role. One effect of importance is the modification of the spin lifetime $T_1$ through dipolar coupling with other NV centers \cite{Jarmola_temperature_2012,mrozek_longitudinal_2015,choi_depolarization_2017}. This particular effect has been used with a levitating diamond in a Paul trap to observe a resonant change in the spins magnetic susceptibility \cite{pellet2021magnetic}.
	
	Other collective effects between NV centers include the cooperative enhancement of the dipole force of the NV centers, a phenomenon similar to that of supperradiance described in \cite{Bachelard,PANAT,Venkatesh}, and observed with a levitating diamond in an optical tweezer by Juan et al. \cite{Juan}.
    
    \bibliographystyle{plain}
	\bibliography{Biblio_ensemble}
    
  \end{document}	
  
  