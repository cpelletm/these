\documentclass{article}
\usepackage{physics}
\usepackage[]{quoting}

\begin{document}


\section*{Summary of the changes}

We would first like to thank the referees for the time they spent reviewing of our manuscripts.

[Résumer la position de ref et dire qu'ils ont raison meme si c'est pas vrai]

We agree with these remarks and have amended the manuscript in consequence to the best of our abilities.

\section*{Referee 1}
\subsection*{Referee 1 report}

\begin{quote}
This manuscript describes experiments showing mechanically detected
cross-relaxation resonance of nitrogen-vacancy centers in diamond. The
measurements are carried out on NV-centers hosted in nano-diamonds
levitated in a Paul trap in vacuum. The authors motivate their work as
a means of detecting paramagnetic defects without optical transitions,
for investigating spin relaxation processes, and for cooling the
motion of mechanical oscillators.

The work appears valid and the interpretation of the measurements and
the accompanying analysis appears well-founded. Although the
presentation is sometimes difficult to follow, the authors have
included an exhaustive supplementary section spelling out the details
of the experiment.

Nevertheless, the importance and broad interest of the work --
requirements for publication in PRL -- appear to this reader to fall
below threshold. I am hard pressed to identify how this work,
"substantially advance[s] a field, open[s] a significant new area of
research, or solves[s]—or take[s] a crucial step toward solving—a
critical outstanding problem". Cross-relaxation is a well-known
phenomenon, also in NV centers. Although it appears not to have been
studied in nano-diamonds nor to have been detected mechanically, it is
not clear why these achievements solve an outstanding problem of
interest.

The authors suggest in the conclusion that the scheme can be used to
cool a nano-diamond in the Paul trap, though no such scheme is
attempted in the paper. Also, no estimates are given as to how
efficient this process would be and what the minimum achievable
temperature would be. The second motivation offered is that this
method could be used as a spectroscopic technique for sensing dipolar
interaction between NV centers and spins that are not optically
active. Again, no such demonstration experiment is presented, although
it's not clear why this should not be possible. The final motivation
given is for "bottom-up investigations of magnetism". Unfortunately,
from the description that follows this statement, it's hard to
envision what, specifically, is intended.

For these reasons, it is my opinion that the authors should resubmit
to PRB or a specialized journal of magnetic resonance.
\end{quote}

\subsection*{Our answer}
Here is our answer to the points raised by Referee 1 :

Je sais pas trop quoi répondre sachant qu'il ne parle que de la conclusion. Est-ce qu'on change la conclu (la partie bottom-up magnetism, j'avoue que je vois pas bien ce que ça cible) ? Est-ce qu'on dit qu'on a pas réussi les applications ? Sur la partie détection de spin au moins je peux justifer : les seuls autres spins qu'on sache être présent ici c'est les P1 et non seulement c'est galère d'aligner précisement un champ mag sur la 111, en plus les effets de para/diamagnétisme des NV commencent à être significatifs à 500G.

\section*{Referee 2}
\subsection*{Referee 2 report}
\begin{quote}
In this work the authors investigate using dipolar interactions
between NV centers to enhance magnetic torque and rotate
electrically-charged, paramagnetic diamond nanocrystals that are
levitated in an ion trap, through a mechanism similar to the
Einstein-de-Haas effect for ferromagnets. An external magnetic field
is tuned to produce resonant cross-relaxation, resulting in greater
paramagnetism and spin torque. Prospects for using this approach to
sense dipolar interactions between NV centers and spins that cannot be
polarized optically are interesting. Using NV centers in levitating
diamonds for emulating magnetism at the 100 $\mu_B$ level is another
potentially useful application. I feel this work is novel and
interesting however the paper requires clarifications before I can
recommend publication.

Fig 3: the caption describes “a) and b) show the torque with and
without cross-relaxation between NV centers respectively.” should this
description be reversed? I thought b) was with CR.

The experimental setup figure in the supplemental material is showing
one APD, but the figure in paper shows 2 detectors: APD1 and APD2.
This should be consistent.

A related question: how is the actual angular motion determined and
calibrated? I find the only explanation of how the angle is measured
on pg 10 of the supplemental material “focus a small area of the
speckle image onto an optical fiber and detect the photons transmitted
through the fiber with the APD1. The detected signal is then highly
sensitive to the particle position and orientation.” How exactly is it
sensitive and how is the calibration determined? This should be better
explained in the main text as well as the Supplemental Material.

The work described in the supplemental material Fig. 3 regarding
measuring the PL is done with a static micro diamond. How is the setup
arranged as compared to the work in the Paul trap.

I was surprised there was not even more drift of the orientation of a
trapped nanodiamond in the Paul trap. Can the authors account for the
physics of the drift, in the SI a suggestion is made “The most likely
origin of this drift is the loss of charges of the diamond due to
photoionization by the laser, which changes the trapping conditions
over time.” Can this somehow be made more quantitative? It seems like
for better understanding and controlling the mechanical rotation for
future applications this may be important to understand?
\end{quote}
\subsection*{Our answer}
Here is our answer to the points raised by Referee 2 :

We have inverted the description of figure 3 in the main text to accurately describe the figure. (A FAIRE !j'ai pas le fichier à jour du main text)

We have changed the experimental setup figure in the supplementary material to be consistent with the one in the main text

La calibration de l'angle : Je pense qu'on peut rediriger vers le SI du nature de Tom (il trouve une précision de 0.3 mrad/racine(Hz). Par contre je sais pas dans quel mesure on développe la réponse.

The static experiments have been performed in the same experimental setup, with the diamonds deposited on the trap electrode/antenna instead of levitating inside it.

Le drift : je vois pas trop quoi dire

\end{document}