\documentclass[a4paper,11pt]{report}
\usepackage[]{amsmath}
\usepackage[]{physics} % \bra, \ket etc
\usepackage{graphicx} %Pour les figures je crois
\usepackage{hyperref}
\usepackage[
    backend=biber, 
    natbib=true,
    style=numeric,
    sorting=none, %Pour faire apparaitre les refs dans l'ordre
    hyperref=true
]{biblatex} %Imports biblatex package
\addbibresource{Bib_ch3.bib} %Import the bibliography file

\usepackage{amssymb} %quelques symboles dont gtrsim /lesssim
\usepackage{subcaption} % package pour faire des subfigures
\usepackage{multirow} % package pour multirow/multicolumn
\usepackage{booktabs} % package pour top/mid/bottom rule
\usepackage{tcolorbox} % toujours plus de boites
\usepackage{xcolor} % Pour avoir des couleurs dans les équations

\title{}
\begin{document}
\chapter{Appendix: Computation of $\bar \eta$}
We will discuss here the specifics to the computation of the  $\bar \eta$ factor for different geometric configurations, and how to compute the $T_1$ times in table [REF] of chapter [REF], and table [REF] of chapter [REF].

As a reminder, $\bar \eta$ is defined as the averaged value of $\eta$ over all possible configurations.

\begin{equation}
\bar \eta = \int \rm{Prob}(\eta) \abs{\eta} d\eta,
\end{equation}

\begin{equation}
\eta^2=\frac{1}{3} (\abs{g}^2+\abs{h}^2)  \frac{4\gamma_f^2}{(\omega_f - \omega_{NV})^2+4\gamma_f^2},
\end{equation}

\section{$\bar \eta$}


\printbibliography
\end{document}