\documentclass[aps,twocolumn,showpacs]{revtex4-1}


\usepackage{epsfig}
\usepackage{amsfonts}
\usepackage{amssymb}
\usepackage{mathrsfs}
\usepackage{theorem}
\usepackage{amsmath}
\usepackage{times}
\usepackage{color}
\usepackage[french]{babel}
\usepackage[T1]{fontenc}
\usepackage{ifpdf}
\ifpdf
\usepackage{epstopdf}   
\usepackage{url}
\fi


\begin{document}
\title{Sub-GHz linewidths ensembles of SiV centers in a diamond nano-pyramid revealed by charge state conversion}

\author{Cl{\'e}ment Pellet-Mary$^1$, Louis Nicolas$^1$, Tom Delord $^1$, Paul Huillery $^1$, and Gabriel H{\'e}tet$^{1}$}
\email{clement.pellet-mary@phys.ens.fr}
\affiliation{$^1$ Laboratoire de Physique de l'Ecole normale sup{\'e}rieure, ENS, Universit{\'e} PSL, CNRS,Sorbonne Universit{\'e}, Universit{\'e} Paris-Diderot, Sorbonne Paris Cit{\'e}, Paris, France
}

\begin{abstract}
\normalsize
\noindent
Efficient interfaces between two-level emitters and photons are key components of quantum networks and sensing. %Diamond negatively charged silicon vacancy centers (SiV$^-$) can be used for such applications as they are photostable and reveal a large emission in the zero phonon line. Nanostructures are required in order to avoid reflections to ensure a strong coupling between the emitters and the light field. 
SiV$^-$ centers embedded into a diamond nanostructure are promising for such purpose. We have studied photoluminescence properties at cryogenic temperatures of diamond AFM probes [1,2] and observed photochromism of a very dense ensemble of SiV$^-$ centers consisting in trapping into a dark state under resonant excitation when nitrogen impurities are present. We suppose that this effect is a charge state switching between negatively charged and neutral SiV charge states.  This effect is used to perform persistent hole burning which reveals very low homogeneous broading, only twice the lifetime limit. It is promising  for quantum optics experiments and paves the way to sub-wavelength microscopy technics. %such as ground state depletion microscopy

We will also present our latest results on the Zeeman splitting and possible lifetime measurement of the SiV$^-$ $1/2$ spin.

\includegraphics[width=.7\columnwidth]{fig_hyp}\\
\noindent\textbf{Figure 1.} a) Averaged PLE measured at 6K by scanning the resonant laser at 737 nm around an electronic transition transition in the presence of green laser light between scans. b) PLE spectra after three different exposure times at 6K : in black without hole burning, in blue t$_{exp}$ = 300 ms and in orange, t$_{exp}$ = 1500 ms. The dots represent raw data and the continuous lines are fitted data. Here P$_{737}$= 0.877 $\mu$W. The inset shows the width of the holes $\Gamma_h$ inferred from the fit and normalized to the homogeneous linewidth $\Gamma_e$(W= $\Gamma_h/\Gamma_e$).

\end{abstract}

\maketitle



\begin{thebibliography}{}

\bibitem{Ref1} AR. Nelz et al., Applied Physics Letters, 109, 19, 193105, (2016).

\bibitem{Ref2} L. Nicolas et al., AIP Advances, 8, 6, 065102, (2018).

\end{thebibliography}
\end{document}
