\documentclass[a4paper,11pt]{article}
\linespread{1.1}
\usepackage[top=3cm, bottom=2.54cm, left=2.54cm, right=2.54cm]{geometry}
\usepackage{amsmath}
\usepackage{amsfonts}
\usepackage{amssymb}
\usepackage{graphicx}

\begin{document}
\pagenumbering{gobble}

\Large
 \begin{center}
Sub-GHz linewidths ensembles of SiV centers in a diamond nano-pyramid revealed by charge state conversion

\hspace{10pt}

\normalsize
% Author names and affiliations
Cl{\'e}ment Pellet-Mary $^1$, Louis Nicolas $^1$, Tom Delord $^1$, Paul Huillery $^1$, Gabriel H{\'e}tet$^1$\\

\hspace{10pt}

\small  
$^1$ Laboratoire de Physique de l'Ecole normale sup{\'e}rieure, ENS, Université PSL, CNRS,Sorbonne Universit{\'e}, Universit{\'e} Paris-Diderot, Sorbonne Paris Cit{\'e}, Paris, France.\\


\end{center}

\hspace{10pt}

\normalsize

\noindent
Efficient interfaces between two-level emitters and photons are key components of quantum networks and sensing. %Diamond negatively charged silicon vacancy centers (SiV$^-$) can be used for such applications as they are photostable and reveal a large emission in the zero phonon line. Nanostructures are required in order to avoid reflections to ensure a strong coupling between the emitters and the light field. 
SiV$^-$ centers embedded into a diamond nanostructure are promising for such purpose. We have studied photoluminescence properties at cryogenic temperatures of diamond AFM probes [1,2] and observed photochromism of a very dense ensemble of SiV$^-$ centers consisting in trapping into a dark state under resonant excitation when nitrogen impurities are present. We suppose that this effect is a charge state switching between negatively charged and neutral SiV charge states.  This effect is used to perform persistent hole burning which reveals very low homogeneous broading, only twice the lifetime limit. It is promising  for quantum optics experiments and paves the way to sub-wavelength microscopy technics. %such as ground state depletion microscopy

We will also present our latest results on the Zeeman splitting and possible lifetime measurement of the SiV$^-$ $1/2$ spin. 

%In our search for a suitable sample, we have studied in detail and at cryogenic temperature sharp single crystal diamond AFM probes from Artech Carbon. Due to the growth conditions, a lot of SiV centers are present at the apex of such a tip whose radius of curvature is as low as 10 nm [3]. They exhibit remarkable optical properties at low temperature.
%We have performed measurements with a scanning Fabry-P{\'e}rot interferometer [4]. This way, we have shown that the inhomogeneous linewidth of the ensemble of emitters is 15 GHz at 6 K. That means that the structure has a very low strain which is probably due to the slow and controlled CVD growth along the $\langle001\rangle$ axis [5].

%Moreover, when nitrogen impurities are also present, SiV centers are trapped into a dark state under resonant excitation. Its lifetime is very long (more than hours). Illumination with a green laser is needed to recover the photoluminescence. We suppose that this effect is a charge state switching between negatively charged and neutral SiV charge states [6]. 
%The charge transfer dynamics enables persistent spectral hole burning which unveils the homogeneous width as shown in Figure 1. Thanks to a rate model, we estimate the homogeneous linewidth $\Gamma_e$ to be 390 MHz at 6 K [7].


%Figure
\hspace{10pt}
\begin{figure}[h!]\centering
\includegraphics[width=.7\columnwidth]{fig_hyp}\\
\noindent\textbf{Figure 1.} a) Averaged PLE measured at 6K by scanning the resonant laser at 737 nm around an electronic transition transition in the presence of green laser light between scans. b) PLE spectra after three different exposure times at 6K : in black without hole burning, in blue t$_{exp}$ = 300 ms and in orange, t$_{exp}$ = 1500 ms. The dots represent raw data and the continuous lines are fitted data. Here P$_{737}$= 0.877 $\mu$W. The inset shows the width of the holes $\Gamma_h$ inferred from the fit and normalized to the homogeneous linewidth $\Gamma_e$(W= $\Gamma_h/\Gamma_e$).
\end{figure}

%This effect reveals that the linewidth is close to be lifetime limited as required for quantum optics experiments and paves the way to sub-wavelength microscopy technics such as ground state depletion microscopy [8]. 
%
%We will also present our latest results on the modification of lifetime of solid state emitters in a half-cavity set-up.

%References
\hspace{10pt}

\noindent{[}1{]} R. Nelz et al., Applied Physics Letters, 109, 19, 193105, (2016).\\
\noindent{[}2{]} L. Nicolas et al., AIP Advances, 8, 6, 065102, (2018).\\
\noindent{[}2{]} L. Nicolas et al., AIP Advances, 8, 6, 065102, (2018).\\
\end{document}


