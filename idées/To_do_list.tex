\documentclass[a4paper]{report}
\usepackage[utf8]{inputenc}
\usepackage[]{amsmath}
\usepackage[]{braket} % \bra, \ket etc
\usepackage{graphicx}
\usepackage{tikz}
\usepackage{subcaption} % package pour faire des subfigures
\usepackage{multirow} % package pour multirow/multicolumn
\usepackage{booktabs} % package pour top/mid/bottom rule
\usetikzlibrary{optics}
\usetikzlibrary{shapes}
\usetikzlibrary{fit}

\title{Titre}
\author{Clément Pellet-Mary}
\date\today

\begin{document}
\chapter{La liste que je regarderai jamais}
  \section{Théorie}
  \begin{itemize}
  \item Changement de référentiel, représentation de Heisenberg etc...
  \item Spins : additions \& co
  \item Règles de sélection
  \item ESR, CPT et co
  \item origine du dipole de transition, lien avec la polarisibilité etc.
  \item un peu d'optics f2f
  \end{itemize}
  \section{Questions centre NV}
  \begin{itemize}
  \item Phonon side-band
  \item mieux comprendre le D
  \item inter-system crossing + Jahn-Teller effect
  \item dependence du spin en E, effet stark, et pk la suceptibilité parallèle est 50x plus petite que la perpendiculaire (Yao)
  \item dépendence du spin en le strain, en la température
  \end{itemize}
 
  \end{document}	
  
  