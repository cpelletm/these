\documentclass[a4paper]{report}
\usepackage[]{amsmath}
\usepackage[]{braket} % \bra, \ket etc
\usepackage{graphicx}
\usepackage{tikz}
\usepackage{subcaption} % package pour faire des subfigures
\usepackage{multirow} % package pour multirow/multicolumn
\usepackage{booktabs} % package pour top/mid/bottom rule
\usetikzlibrary{optics}
\usetikzlibrary{shapes}
\usetikzlibrary{fit}

\title{Titre}
\author{Clément Pellet-Mary}
\date\today

\begin{document}
\chapter{La liste que je regarderai jamais}
  \section{Théorie}
  \begin{itemize}
  \item Changement de référentiel, représentation de Heisenberg etc...
  \item Spins : additions \& co
  \item Règles de sélection
  \item ESR, CPT et co
  \item origine du dipole de transition, lien avec la polarisibilité etc.
  \item un peu d'optics f2f
  \end{itemize}
  \section{Questions centre NV}
  \begin{itemize}
  \item Phonon side-band
  \item mieux comprendre le D
  \item inter-system crossing + Jahn-Teller effect
  \item dependence du spin en E, effet stark, et pk la suceptibilité parallèle est 50x plus petite que la perpendiculaire (Yao)
  \item dépendence du spin en le strain, en la température
  \end{itemize}
 
 \section{Manips}
 \begin{itemize}
 \item Dépendance polarization time/distance objectif (micro/bulk)
 \item Le T1 soustraction dépend de la distance de l'objectif : dépend aussi de la puissance du laser ? Juste pour le T1 dipole ou aussi le T1 phonon ?
 \item Mesurer des largeurs fluctateurs avec le T1 sur plein d'échantillons (de Ludo au rose) (et vérifier que PL=T1 si possible, mais ça rajoute des contraintes pour rien...).
 \item Scan de pseudo-T1 (juste 2 points : S1 au début, S2 au milieu et (S1-S2)/S1 ) en fonction de la fréquence. Je viens de réfléchir et c'est pas si malin

 \item faire une courbe sensibilité (magnetometry) en fonction de l'amplitude du champ mag modulé, selon la 100 et selon une 1x1x1x1
 \item Mettre en place le protocole de magnetometrie par feedback (avec pyrpl probablement) puis faire une carte, a terme sur des 1/10 um sur un circuit imprimé.
 \item La dépendance PL en fonction du champ transverse est pas la meme pour des échantillons denses et pas dense ? Vérfier que c'est pas une blague de l'APD-compteur-de-photon mesurée en tension. Peut-etre aussi une blague de filtrage spatial avec la fibre.
 \item les différentes phases pour le T1 dipolaire, l'ESR et le champ transverse.
 \item effet de la polarisation du laser (après le dichroique) sur le contraste ESR (quantifier la polarisation de chaque classe si possible, même si je controlle pas à 3D), puis sur le T1 de spin. (cf http://dx.doi.org/10.1088/1367-2630/17/2/023040)
 \item Les coefs beta des stretch exponential (en fittant) ont l'air d'être modifié que par les double flips/base couplée (sur /these/data/20220324/Adamas 150 um/adamas 2). A vérifier si c'est le cas pour un croisement 121 et pour le champs transverse ou juste un hasard. 
 \item test magnetometry en mode 2x2 proches (genre sur le coté de la lorentzienne). p-e avec des bobines de Helmotz + aimant permanent. en faite je sais pas d'ou il faudrait partir, ce serait peut-etre plus simple juste sur une 121 pour commencer.
 \end{itemize}
 
 \section{Done}
 \begin{itemize}
  \item le 12C et 15N en champ transverse n'a qu'une bosse (normal) et la bosse du + et beaucoup plus petite que la bosse du -, et ca c'est pas normal. Ca vient vraiment du champ transverse vu que ça dépend beaucoup de l'angle et peu de l'amplitude du champ mag (donc des fréquences : pas un pb de micro-onde)
  
  $\to$ C'est un problème de polarisation de la micro-onde : en champ transverse (E ou B d'ailleurs) les transitions sont senseibles à une polarisation longitudinale de la micro-onde : Sx pour $\ket{0} \to \ket{+}$ et Sy pour $\ket{0} \to \ket{-}$ par exemple, alors qu'en champ mag longitudinal, elles sont sensibles à la polarisation circulaire. Comme ma micro-onde est polarisée linéairement mais pas circulairement, je vois une différence qu'en champ transverse.
  
   \item T1 soustraction en fonction de la longueur d'onde de la micro-onde (à faible puissance du coup) en champ nul. Potentiellement en 121 a moitié splitté aussi.
   
   $\to$ +/- done, j'ai surtout l'impression que ça suit le contraste ODMR. Faudrait que je vérifie qu'il y ait pas des bugs dans mon programme, surtout sur la procédure de fit (T1 plus faibles pour un signal moins bon).
 \end{itemize}
  \end{document}	
  
  