\documentclass[a4paper]{article}
\usepackage[]{amsmath}
\usepackage[]{physics} % \bra, \ket etc
\usepackage{graphicx} %Pour les figures je crois
\usepackage[
    %backend=biber, 
    natbib=true,
    style=numeric,
    sorting=none %Pour faire apparaitre les refs dans l'ordre
]{biblatex} %Imports biblatex package
\addbibresource{Bib_ch2.bib} %Import the bibliography file

\usepackage{subcaption} % package pour faire des subfigures
\usepackage{multirow} % package pour multirow/multicolumn
\usepackage{booktabs} % package pour top/mid/bottom rule
\usepackage{tcolorbox} % toujours plus de boites
\usepackage{xcolor} % Pour avoir des couleurs dans les équations

\title{Appendix A : Dipole-dipole interaction between two spins in solid}
\begin{document}
The dipole-dipole interaction between two point-like magnetic dipoles can be derived classically by computing the static magnetic field generated by one dipole at the position of the second dipole.

For a magnetic dipole \textbf{m}, positioned on \textbf{0}, the static magnetic field generated on position $\mathbf{r}=r\mathbf{u}$ is : \cite[p.~188]{jackson1999classical} 
\begin{equation}
\mathbf{B}(\mathbf{r})=\frac{\mu_0}{4 \pi}\left[ \frac{3 (\mathbf{m}\cdot\mathbf{u})\mathbf{u} - \mathbf{m}}{r^3}+\frac{8\pi}{3}\mathbf{m}\,\delta(\mathbf{r})\right]
\end{equation}

The interaction energy between two dipole $\mathbf{m}_1$ and $\mathbf{m}_2$ with relative position \textbf{r} is therefore :

\begin{equation}
U=-\mathbf{m}_1 \cdot \mathbf{B}_2(\mathbf{r})=-\frac{\mu_0}{4 \pi}\left[ \frac{3 (\mathbf{m}_1\cdot\mathbf{u})(\mathbf{m}_2\cdot\mathbf{u}) - \mathbf{m}_1\cdot\mathbf{m}_2}{r^3}+\frac{8\pi}{3}\mathbf{m}_1\cdot\mathbf{m}_2\,\delta(\mathbf{r})\right]
\end{equation}

The term in $\delta$, which comes from the necessity to preserve $\div{\vb B}=0$, is at the origin of the Fermi contact energy. This term plays a role for hyperfine coupling when there is an overlap of the electron wave-function with the nucleus, which is the case for s-orbitals in atoms. In the case of NV centers, Fermi contact energy plays an important role to the hyper-fine coupling between the NV$^-$ electron spin and the nuclear spin of the $^{14}$N or $^{15}$N atom \cite{doherty2012theory} or for nearby $^{13}$C atoms \cite{smeltzer201113c}.

However, in the scope of this manuscript, we will mostly concern ourselves with dipolar interaction between electronic spins separated by several atomic sites, so that the overlap of the electronic wave functions can be mostly neglected. This results in the simplified dipole-dipole coupling energy :

\begin{equation}
\label{eq_dipole_dipole_classical}
U=\frac{\mu_0}{4 \pi}\left[ \frac{ \mathbf{m}_1\cdot\mathbf{m}_2 - 3 (\mathbf{m}_1\cdot\mathbf{u})(\mathbf{m}_2\cdot\mathbf{u})}{r^3}\right]
\end{equation}

For two electronic spins, we can rewrite a quantized version of \ref{eq_dipole_dipole_classical} by replacing the magnetic moments $\mathbf{m}_i=\hbar \gamma_i \hat{\mathbf{S}}_i$ where $\gamma_i=-\frac{g_e \mu_B}{\hbar}$ is the gyromagnetic ratio of the electron. For the NV$-$ electronic spin, as well as most electronic spin defects in diamond, $g_e\approx 2$ which gives a numerical value for the gyromagnetic ratio $\gamma \approx (2\pi)\,28\ \rm{GHz/T} \approx (2\pi)\,2.8\ \rm MHz/G$. $\hat{\mathbf{S}}_i=(\hat S_x, \hat S_y, \hat S_z)$ is the spin vector operator for particle $i$. The quantized version of the dipole-dipole interaction is then:

\begin{equation}
\mathcal{H}_{dd}=\frac{J_0}{r^3}\left[ \hat{\mathbf{S}}_1\cdot\hat{\mathbf{S}}_2 - 3 (\hat{\mathbf{S}}_1\cdot\mathbf{u})(\hat{\mathbf{S}}_2\cdot\mathbf{u})\right]
\end{equation}

Where $J_0\equiv \frac{\mu_0\gamma_1\gamma_2 \hbar^2}{4\pi}$. For two electronic spins with $g-$factors close to 2, the numerical value of $J_0$ is $(2\pi)\,52\ \rm{MHz}\cdot\rm{nm}^3$
\printbibliography
\end{document}