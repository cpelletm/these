\documentclass[a4paper]{report}
\usepackage[utf8]{inputenc}
\usepackage[]{amsmath}
\usepackage[]{braket} % \bra, \ket etc
\usepackage{graphicx}
\usepackage{tikz}
\usepackage{subcaption} % package pour faire des subfigures
\usepackage{multirow} % package pour multirow/multicolumn
\usepackage{booktabs} % package pour top/mid/bottom rule
\usepackage{tcolorbox} % toujours plus de boites
\usetikzlibrary{optics}
\usetikzlibrary{shapes}
\usetikzlibrary{fit}

\title{Titre}
\author{Clément Pellet-Mary}
\date\today

\begin{document}
\chapter{Fiche de lecture des articles}
  \section{Nano-pyramides Louis 2019}
  \subsection{Description des échantillons}
  Les échantillons c'est des pointes d'AFM de $\approx 15 \mu$m de long et $\leq 10$ nm de large au niveau de la pointe et qui, à cause de leur synthèse sur un substrat de silicium, ont des centres SiV, particulièrement au niveau de la pointe.
  
  Du coup, on peut focus sur une pointe unique, et dans ces conditions tu as très peu d'élargissement inhomogène ($\approx 10$ GHZ), ce qui est inférieur à la structure fine du SiV qui est de 50 et 250 GHZ. Donc tu vois les 4 pics au spectro.
 
  \end{document}	
  
  