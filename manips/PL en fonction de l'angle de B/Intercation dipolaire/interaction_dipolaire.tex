\documentclass[a4paper]{report}
\usepackage[utf8]{inputenc}
\usepackage[]{amsmath}
\usepackage[]{braket} % \bra, \ket etc
\usepackage{graphicx}
\usepackage{tikz}
\usepackage{subcaption} % package pour faire des subfigures
\usepackage{multirow} % package pour multirow/multicolumn
\usepackage{booktabs} % package pour top/mid/bottom rule
\usepackage{tcolorbox} % toujours plus de boites
\usetikzlibrary{optics}
\usetikzlibrary{shapes}
\usetikzlibrary{fit}

\title{Titre}
\author{Clément Pellet-Mary}
\date\today

\begin{document}
\chapter{Interaction dipolaire}
  \section{Hamiltonien général}
  Le hamiltonien général est :
  \begin{equation}
  \hat{\mathcal{H}}_{dd}= \dfrac{\mu_0}{4\pi}\gamma_1\gamma_2\hbar^2\dfrac{1}{r^3}(\hat{\mathbf{S_1}} \cdot \hat{\mathbf{S_2}} - 3(\hat{\mathbf{S_1}} \cdot \mathbf{r})(\hat{\mathbf{S_2}} \cdot \mathbf{r}))
  \end{equation}
  \begin{center}
  Avec $gamma_i$ le rapport gyromagnétique du spin i et $J_0=\dfrac{\mu_0}{4\pi}\gamma_1\gamma_2\hbar^2=52\, \rm MHz / nm^3$ pour l'interaction $\rm NV-NV$
\end{center}   
  
  \section{Réécriture pour deux spins 1}
  La tout de suite j'ai la flemme de me taper des heures de latex, en gros il faut se baser sur le supplementary de l'article "critical thermalization" de Lukin, il y a trois sortes de termes :
  \begin{
 
  \end{document}	
  
  