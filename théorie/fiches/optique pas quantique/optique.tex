\documentclass[a4paper]{report}
\usepackage[utf8]{inputenc}
\usepackage[]{amsmath}
\usepackage[]{braket} % \bra, \ket etc
\usepackage{graphicx}
\usepackage{tikz}
\usepackage{subcaption} % package pour faire des subfigures
\usepackage{multirow} % package pour multirow/multicolumn
\usepackage{booktabs} % package pour top/mid/bottom rule
\usepackage{tcolorbox} % toujours plus de boites
\usetikzlibrary{optics}
\usetikzlibrary{shapes}
\usetikzlibrary{fit}

\title{Titre}
\author{Clément Pellet-Mary}
\date\today

\begin{document}
\chapter{Optique pas quantique}
  \section{Ordres de grandeurs}
  \subsection{Force électrique Vs Force magnétique}
  La raison pour laquelle on s’intéresse par défaut plus au champ électrique que magnétique, c'est parce que le rapport des deux termes dans la force de Lorentz vaut :\begin{equation}
  \dfrac{F_E}{F_B}=\dfrac{q\textbf{E}}{q\textbf{v}\times \textbf{B}}=\dfrac{c}{v}
  \end{equation}
  Il se trouve qu'on est pas si loin des vitesse relativistes pour les électrons d'un atome, en effet, en partant du quantum de moment cinétique \begin{equation}
  \hbar=mvr=m_e\,v\,a_0
  \end{equation}
  Avec $m_e$ la masse de l'électron et $a_0$ le rayon de Bohr, que l'on détermine en posant le PFD : $\dfrac{m v^2}{r}=\dfrac{q^2}{4\pi \epsilon_0r^2}$, d'où :\begin{equation}
  a_0=\dfrac{\hbar^2 4\pi\epsilon_0}{m_e q_e^2}\approx 0.53 \mathrm{\AA}
  \end{equation}
  
  Et donc finalement : \begin{equation}
  \dfrac{F_E}{F_B}=\dfrac{c\hbar}{m_e a_0}=\dfrac{1}{\alpha}\approx 137
  \end{equation}
  
  C'est également le ratio des transitions Dipolaire électrique Vs Dipolaire magnétique par défaut.
 
  \end{document}	
  
  