\documentclass[a4paper]{report}
\usepackage[utf8]{inputenc}
\usepackage[]{amsmath}
\usepackage[]{braket} % \bra, \ket etc
\usepackage{graphicx}
\usepackage{tikz}
\usepackage{subcaption} % package pour faire des subfigures
\usepackage{multirow} % package pour multirow/multicolumn
\usepackage{booktabs} % package pour top/mid/bottom rule
\usepackage{tcolorbox} % toujours plus de boites
\usetikzlibrary{optics}
\usetikzlibrary{shapes}
\usetikzlibrary{fit}

\title{Titre}
\author{Clément Pellet-Mary}
\date\today

\begin{document}
\chapter{Introduction}
  \section{The diamond}
  The Diamond is a crystal comprised solely of carbon atoms, each one being bonded to four other atoms through covalent bonds. Because the covalent bonds are the strongest kind of chemical bonds, even more so for light atoms such as carbon, the diamond shows a a great
  
  \section{The NV Center}
  A compléter à un moment,
  
  Hamiltonien de spin simplifié : 
  \begin{equation}
  \hat{\mathcal{H}}_s=D S_z^2 + \gamma_e \textbf{B}\cdot\hat{\textbf{S}}
  \end{equation}
  \begin{center}
  $D=2.87 \,$GHz et $\gamma_e=2.8\,\mathrm{MHz}/\mathrm{G}$
\end{center}   
  
  toto
 
  \end{document}	
  
  