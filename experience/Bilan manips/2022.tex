\documentclass[a4paper]{article}
\usepackage[french]{babel}
\usepackage[utf8]{inputenc}
\usepackage[]{amsmath}
\usepackage[]{braket} % \bra, \ket etc
\usepackage{graphicx}


\title{Bilan manips 2022}

\begin{document}
\maketitle
\section{Adamas}
\begin{itemize}
\item \textbf{01/24} Test sensi en fonction de puissance las sur un bon adamas, ESR 0B et 1x4, scans 100 et 1x4
\item \textbf{02/18} Premiers tests T1 121 vs 22 sur un adamas 15 um. Ca marche
\item \textbf{02/22} Beaucoup de mesure sur un adamas 15 um sur la pcb : des T1 dans tous les sens (dont 121(x4) et 22 (x4), mais aussi 100,111, 1 classe, C13 en champ nul et du champ transverse (petit moyen et grand). Les ESR associés, et une mesure de la largeur des fluctuateurs en T1 (insh'Allah. Pour l'instant ça merdouille pas mal).
\end{itemize}
\section{Echantillons Alex}
\begin{itemize}
\item \textbf{01/18} Quelques tentatives de Rabi sur la face arrière (HPHT) d'un échatillon d'Alex
\item \textbf{01/21} ESR 0B et champ non nul pour différentes poistion de l'échantillon (mieux fait en 2021)
\item \textbf{01/28} Quelques ESR sur le rose dont un ESR 100 avec les 3 raies du spin nucl visibles (mais bcp d'hysteresis de l'autre coté)
\item \textbf{02/21} Test T1 121 vs 22 sur diamant rose, ça marche pas (différence trop faible)
\end{itemize}
\section{divers}
\begin{itemize}
\item \textbf{02/17} Test puissance micro-onde en fonction de la fréquence, avec et sans clapet anti-retour, sur un un ensemble de adamas 1 um.
\item \textbf{02/21} Essai de mesures EPR en transmission avec la PCB et le diamant rose. Ca marche pas. Mais j'ai test en modulant le laser, j'aurais du moduler le champ mag.
\end{itemize}
\end{document}