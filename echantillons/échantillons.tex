\documentclass[a4paper]{report}
\usepackage[utf8]{inputenc}
\usepackage[]{amsmath}
\usepackage[]{braket} % \bra, \ket etc
\usepackage{graphicx}
\usepackage{tikz}
\usepackage{subcaption} % package pour faire des subfigures
\usepackage{multirow} % package pour multirow/multicolumn
\usepackage{booktabs} % package pour top/mid/bottom rule
\usetikzlibrary{optics}
\usetikzlibrary{shapes}
\usetikzlibrary{fit}

\title{Titre}
\author{Clément Pellet-Mary}
\date\today

\begin{document}
\chapter{Échantillons}

  \section{SiV}
  
  \subsection{template}
  \subsubsection{Mesures réalisées}
  \subsubsection{Observations}
  \subsubsection{Rangé}
  
  \subsection{"échantillon d'alex"} 
  C'est un diamant bulk (procédé de fabrication ?) avec une fine couche de SiV près de la surface.
  \subsubsection{Mesures réalisées}
  \begin{itemize}
  \item spectres à chaud et à froid : 2019/09/02 $\to$ 2019/09/12
  \item tentative de PLE : 2019/09/09 $\to$ 2019/09/12
  \end{itemize}
  \subsubsection{Observations}
  Spectre de SiV assez compact (~3 nm de ZPL), la PLE n'a pas réussie à isoler de centre unique, soit parce qu’il y en a trop, soit parce que je me suis loupé.
  
  PL avec le laser vert : max 8E4, PLE avec le laser résonnant : max 4E4.
  \subsubsection{Rangé}
  Dans une boite d'Alex marqué ...Si...Recuit 1100 (pas sur que ça corresponde)
 
 \subsection{Nano-pyramides "Elke Neu"}
  \subsubsection{Mesures réalisées}
  \begin{itemize}
  \item Des spectres à chaud(2019/09/19)
  \item Cf manip pour plus de précisions
  \item 22/10/2019 : du Hole-burning pour des pyramides posées directement sur un aimant néodyme.
  \end{itemize}
  
  
  \subsubsection{Observations}
  Ca crache bien, 6E6 de PL au laser vert et à l'objectif x10 sans forcer, et à travers la plaque de quartz. Par contre le NV est beaucoup plus visible que le SiV.
  
  (15/10/19) En faite les SiV crachent pas mal, 1.3E6 avec les filtre pour la ZPL, 5E5 avec le filtre pour la PSB (PLE)
  \subsubsection{Rangé}
  \begin{itemize}
  \item Solution mère : Dans une boite "gel" marquée "nano-diamant pyramide CVD" (ou un truc du genre) dans le tiroir des échantillons.
  \item Échantillon sur lame de quartz épaisse: sur son porte-échantillon t dans le cryostat.
  \end{itemize}
    
 \section{Rubis}
 \subsection{L'échantillon du stage}
 Le joli, même si je crois que techniquement c'est pas celui des photos au MEB.
 \subsubsection{Mesures réalisées}
 cf le stage, des temps de vie, de la saturation, des spectres ... 
 \subsubsection{Observations}
 Bah pas de modifications de temps de vie pour l'instant
 \subsubsection{Rangé}
 \begin{itemize}
 \item Solution mère dans un tube marqué "rubis" dans un pot avec d'autres tubes, dans le tiroir échantillon (j'espère)
 \item Échantillon qui va bien : sur une fine lame de quartz, sur le porte-échantillon, dans le cryostat.
 \end{itemize}
 
 
 
 \section{Plaques de quartz}
 Plaques de quartz de 3x3 mm nettoyées à l'acétone + IPA et propres au microscope (certaines en tout cas)
 \subsubsection{Rangé}
 Dans une boite "gel" marqué "propre" dans le tiroir des échantillons.
  \end{document}	
  
  